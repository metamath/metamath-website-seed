
\documentclass{slides}

%page size
\usepackage{anysize}
\papersize{8in}{11in}
%left,right,top,bottom
%\marginsize{0.4706in}{0.4706in}{0.476in}{0.4706in}
\marginsize{1in}{1in}{0.2in}{1in}

\usepackage{color}
\definecolor{mint}{rgb}{.933,1,.98}
\definecolor{darkgreen}{rgb}{0,.7,0}

\usepackage{amssymb}
\usepackage{latexsym}
\usepackage{amsthm}

\begin{document}
\raggedright
\pagecolor{mint}


\begin{slide}
\begin{center}
\textcolor{blue}{\textbf{\LARGE Orthomodular Lattices\\
and Beyond}}\\
\vspace{3ex}
{\large Norman Megill}\\
\vspace{1ex}
\textcolor{darkgreen}{\texttt{nm {}@ alum.mit.edu\qquad http://metamath.org}}
\end{center}
\end{slide}

\begin{slide}
An \textcolor{red}{\textit{ortholattice} (OL)} is an algebra
$\langle A,\cup,\cap,'>$ in which the following conditions hold:
\begin{eqnarray}
a\cup b & = & b\cup a\\
(a\cup b)\cup c & = & a\cup (b\cup c)\\
a'' & = & a\\
a\cup (a\cap b) & = & a\\
a\cap b & = & (a'\cup b')'
\end{eqnarray}
An \textcolor{red}{\textit{orthomodular lattice} (OML)} is an OL in which
\begin{eqnarray}
a\cup b & = & ((a\cup b)\cap b')\cup b
\end{eqnarray}


A \textcolor{red}{\textit{Boolean algebra} (BA)}
is an OML in which
\begin{eqnarray}
a & = & (a\cap b)\cup (a\cap b')
\end{eqnarray}
\end{slide}

\begin{slide}
\textcolor{blue}{\textbf{A Neat Result}}

There is a single axiom of length 23 for OML, where
$\textcolor{red}{a|b}\,{\buildrel\rm def\over =}\,
a'\cup b'$ is the Sheffer stroke (McCune, Rose, Veroff
\textcolor{darkgreen}{
\texttt{http://www.mcs.anl.gov/{\char`\~}mccune/papers/olsax/}}):
\begin{eqnarray}
((((b|a)|(a|c))|d)|(a|((c|((a|a)|c))|c))) &=
  & a  % OML-saa
\end{eqnarray}

Open problem(?):  is there one of length 21?
\end{slide}


\begin{slide}

\textcolor{blue}{\textbf{Some Definitions}}
\begin{eqnarray}
\textcolor{red}{1} &
{\buildrel\rm def\over =}& a\cup a' \quad\mbox{(unit)}\\
\textcolor{red}{0} &
{\buildrel\rm def\over =}& 1' \quad\mbox{(zero)}\\
\textcolor{red}{a\le b} &
{\buildrel\rm def\over \Leftrightarrow}& a=a\cap b
 \quad\mbox{(less-than-or-equal)} \\
\textcolor{red}{a\equiv b} &{\buildrel\rm def\over =}&
(a\cup b)\cap(a'\cup b')  \quad\mbox{(equivalence)} \\
\textcolor{red}{a \,{\rm C}\, b} &
{\buildrel\rm def\over \Leftrightarrow}& a=(a\cap b)
 \cup(a\cap b')
 \quad\mbox{(commutes)}
\end{eqnarray}

\end{slide}



\begin{slide}
A \textcolor{red}{\textit{weakly orthomodular lattice} (WOML)}
 is an OL in which
\begin{eqnarray}
(a'\cap (a\cup b))\cup b'\cup (a\cap b) & = & 1
\end{eqnarray}

A \textcolor{red}{\textit{weakly Boolean algebra} (WBA)}
 is a WOML in which
\begin{eqnarray}
a'\cup (a\cap b)\cup (a\cap b') & = & 1
\end{eqnarray}




%\begin{figure}[htbp]\centering
  % Set unitlength to default in case it's not
  \setlength{\unitlength}{1pt}
  \begin{picture}(100,90)(0,0)

    \put(30,0){
      \begin{picture}(60,80)(-10,-10)
%                1
%               / \
%              y   x'
%              |   |
%              x   y'
%               \ /
%                0
        \put(20,0){\line(-1,1){20}}
        \put(20,0){\line(1,1){20}}
        \put(0,20){\line(0,1){20}}
        \put(40,20){\line(0,1){20}}
        \put(0,40){\line(1,1){20}}
        \put(40,40){\line(-1,1){20}}

        \put(20,-5){\makebox(0,0)[t]{$0$}}
        \put(-5,20){\makebox(0,0)[r]{$x$}}
        \put(45,20){\makebox(0,0)[l]{$y'$}}
        \put(-5,40){\makebox(0,0)[r]{$y$}}
        \put(45,40){\makebox(0,0)[l]{$x'$}}
        \put(20,65){\makebox(0,0)[b]{$1$}}

        \put(20,0){\circle*{3}}
        \put(0,20){\circle*{3}}
        \put(40,20){\circle*{3}}
        \put(0,40){\circle*{3}}
        \put(40,40){\circle*{3}}
        \put(20,60){\circle*{3}}

        \put(100,80){\makebox(0,0)[l]{Lattice O6 is an example of a WOML and a WBA.}}
        \put(100,50){\makebox(0,0)[l]{It is non-orthomodular and non-distributive,}}
        \put(100,20){\makebox(0,0)[l]{yet it is a model for both quantum and classical}}
        \put(100,-10){\makebox(0,0)[l]{propositional calculus!}}

      \end{picture}
    } % end of O6 subpicture

  \end{picture}
%  \caption
%    Lattice O6
%  \label{fig:o6mo2}}
%\end{figure}

\end{slide}


\begin{slide}
\textcolor{blue}{\textbf{Summary of WOML and WBA Results}}

\begin{itemize}
\item All OMLs (BAs) are WOMLs (WBAs).
\item Not all WOMLs (WBAs) are OMLs (BAs).
\item Any OML (BA) equation can be represented in WOML (WBA) with
the following mapping:
\begin{center}
\begin{tabular}{c|c}
OML (BA) & WOML (WBA)  \\
\hline
$a=b$ & $a\equiv b=1$
\end{tabular}
\end{center}
\item WOMLs (WBAs) are more general models for quantum (classical)
propositional calculus, than the usual OML (BA) models.
\end{itemize}
\end{slide}


\begin{slide}

\textcolor{blue}{\textbf{The 6 Implications in OMLs}}
\begin{eqnarray}
\textcolor{red}{a\to_0b} &
{\buildrel\rm def\over =}& a'\cup b \quad\mbox{(classical)}\\
\textcolor{red}{a\to_1b} &
{\buildrel\rm def\over =}& a'\cup(a\cap b) \quad\mbox{(Sasaki)}\\
\textcolor{red}{a\to_2b} &
{\buildrel\rm
def\over =}&\ b'\to_1a'  \quad\mbox{(Dishkant)}\\
\textcolor{red}{a\to_3b} &
{\buildrel\rm def\over =}& (a'\cap b)\cup(a'\cap b')\cup
(a\to_1b)                 \quad\mbox{(Kalmbach)}\\
\textcolor{red}{a\to_4b} & {\buildrel\rm def\over =}&
b'\to_3a'                            \quad\mbox{(non-tollens)}\\
\textcolor{red}{a\to_5b} & {\buildrel\rm def\over =}&
(a\cap b)\cup(a'\cap b)\cup(a'\cap b')  \quad\mbox{(relevance)}
\end{eqnarray}

All 6 implications evaluate
to $a\to_0b$ in a Boolean lattice.
$\to_i$, for $i\ne 0$, is called a
\textcolor{red}{{\em quantum}} implication.  $\to_0$ is
called a \textcolor{red}{{\em classical}} implication.

\end{slide}

\begin{slide}
Quantum implications are distinguished by the fact
in an OML they
satisfy the \textcolor{red}{{\it Birkhoff-von Neumann condition}}:
\begin{eqnarray}
a\to_i b=1 & \Leftrightarrow & a\le b, \quad i=1,\ldots,5
\end{eqnarray}

Neat result (Pavi\v ci\'c/Megill, 1998):
\begin{eqnarray}
a\cup b&=&(a\rightarrow_i b)\rightarrow_i(((a
\rightarrow_i b)\rightarrow_i(b \to_i a))\rightarrow_i a)
\end{eqnarray}
holds in any OML for $i=1,\ldots,5$.  This observation lets us to
construct, by adding a constant $0$, an OML-equivalent ``unified''
algebra with an (unspecified) quantum implication as its only
binary operation.  Thus, we can study the properties common to all
quantum implications without a philosophical debate of which is
the ``real'' implication.

\end{slide}

\begin{slide}

\textcolor{blue}{\textbf{Orthoimplication Algebra
$\langle A,.\rangle$ (Abbott, 1976)}}
\begin{eqnarray}
(ab)a & = & a\\
(ab)b & = & (ba)a\\
a((ba)c) & = & ac
\end{eqnarray}
If ``$.$'' is interpreted as $\to_2$, then each equation holds in OML.

Conjecture (completeness):  All such equations
(i.e.\ polynomials in $\to_2$ on each side of equality) that hold in OML
can be proved from this algebra.


\end{slide}


\begin{slide}



\textcolor{blue}{\textbf{Quasi-Implication Algebra
 $\langle A,.\rangle$ (Hardegree, 1981)}}
\begin{eqnarray}
(a b) a & = & a\\
(a b) (a c) & = & (b a) (b c)\\
((a b)(b a)) a & = & ((b a)
     (a b)) b
\end{eqnarray}
If ``$.$'' is interpreted as $\to_1$, then each equation holds in OML.

Theorem (completeness) [Hardegree]:  All such equations that hold in OML,
 with $\to_1$ as the
only operation, can be proved from this algebra.


\end{slide}

\begin{slide}
\textcolor{blue}{\textbf{Other Implication Algebras}}

Similar algebras for $\to_3$, $\to_4$, and $\to_5$ have not been
proposed nor any completeness results obtained.

The most promising
system for future study is $\to_5$, because  $a\to_1 b=
a\to_5 (a\to_5 b)$ in any OML, holding promise that
the ideas in Hardegree's $\to_1$
proof can be adapted for $\to_5$.

Systems for $\to_3$ and $\to_4$, as well as completeness of
Abbott's $\to_2$ system, remain complete mysteries.

\end{slide}

\begin{slide}

\textcolor{blue}{\textbf{Another Open OML Problem}}

Problem:  does the following equation hold in all OMLs?
\begin{eqnarray}
\lefteqn{(a \to_5 b) \cap (b \to_5 c)
      \cap (c \to_5 d) \cap (d \to_5 e) \cap (e \to_5 a)
          =}\qquad\qquad\nonumber\\
  & & (a \equiv b) \cap (b \equiv c) \cap (c \equiv d) \cap (d \equiv e)
      \label{eq:theeqi5}
\end{eqnarray}
Note:  It holds for $\le 4$ variables.  It does not hold for $\ge 6$ variables.
\end{slide}





\begin{slide}
\textcolor{blue}{\textbf{Orthomodular Lattices and Hilbert Space}}

\textbf{Fact:} The OML axioms hold in the lattice of closed subspaces of
infinite dimensional Hilbert space,
\textcolor{red}{${\cal C}({\cal H})$}.  This is a primary motivation for
studying them.  But they aren't the only equations that hold!

\textbf{Some history:}
\begin{itemize}
\item 1936 - Birkhoff/von Neumann attempt to find a ``logical structure''
  for quantum mechanics, but find only the modular law (holding only
  for finite-dimensional Hilbert space).
\item 1937 - Husumi discovers the orthomodular law and
  shows that it holds in ${\cal C}({\cal H})$.
\end{itemize}

\end{slide}
\begin{slide}
\textcolor{blue}{\textbf{History (cont.)}}

\begin{itemize}
\item 1975 - Day discovers the orthoarguesian law and shows that it
  holds in ${\cal C}({\cal H})$.
\item 1981 - Godowski discovers an infinite equational variety
derived from properties of
 states on ${\cal C}({\cal H})$, that holds
in ${\cal C}({\cal H})$.
\item 1985 - Mayet extends Godowski's discovery to prove the existence
of a more general equational variety that
  holds in ${\cal C}({\cal H})$.  (However Mayet
provides no actual examples of these new equations that are stronger than
Godowski's.)
\end{itemize}

\end{slide}

\begin{slide}


\textcolor{blue}{\textbf{History (cont.)}}

\begin{itemize}
\item 1995 - Sol\`er proves that an OML, with certain additional conditions,
determines a Hilbert space (very significant).  Thus OML theory (with
these conditions) and
Hilbert space theory are duals.
\end{itemize}

\end{slide}

\begin{slide}

\textcolor{blue}{\textbf{Some recent results:}}

\begin{itemize}
\item 2000 - Megill/Pavi\v ci\'c found an infinite equational variety
related to orthoarguesian equations, but stronger, that
holds in ${\cal C}({\cal H})$.
\item 2003 - Megill/Pavi\v ci\'c found examples (unpublished)
 of Mayet's equations
that are stronger than Godowski's, that
hold in ${\cal C}({\cal H})$.
\end{itemize}


\end{slide}

\begin{slide}
\textcolor{blue}{\textbf{Equations Related to States
That Hold in ${\cal C}({\cal H})$}}

The simplest \textcolor{red}{\em Godowski equation} is
\begin{eqnarray}
(a\to_1 b)\cap(b\to_1 c)\cap(c\to_1 a) &\le& a\to_1 c
\end{eqnarray}

Using Mayet's theory, Megill/Pavi\v ci\'c (unpublished) found examples
of equations stronger than (independent from) Godowski's.  The
simplest example is
\begin{eqnarray}
( ( a \to_1 b ) \to_1 ( c \to_1 b ) ) \cap
                     ( a \to_1 c ) \cap ( b \to_1 a )& \le&  c \to_1 a
\end{eqnarray}
\end{slide}

\begin{slide}
\textcolor{blue}{\textbf{More Definitions}}
\begin{eqnarray}
\textcolor{red}{a{\buildrel c\over\equiv}b}
&{\buildrel\rm def\over =}& ((a\to_1 c)
\cap(b\to_1 c))
\cup((a'\to_1 c)\cap(b'\to_1 c)) \\
\textcolor{red}{a{\buildrel c,d\over\equiv}b}
&{\buildrel\rm def\over =}& (a{\buildrel d\over\equiv}b)\cup
((a{\buildrel d\over\equiv}c)\cap
(b{\buildrel d\over\equiv}c))
\end{eqnarray}

\textcolor{blue}{\textbf{Orthoarguesian Equations
That Hold in ${\cal C}({\cal H})$}}
\begin{eqnarray}
(a\to_1 c)\cap (a{\buildrel c\over\equiv}b) & \le  b\to_1 c
\quad\mbox{(\textcolor{red}{OA3})} \\
(a\to_1 d)\cap (a{\buildrel c,d\over\equiv}b) & \le  b\to_1 d
\quad\mbox{(\textcolor{red}{OA4})}
\end{eqnarray}
OA4 is a 4-variable equivalent to Day's original 6-variable orthoarguesian
equation.  OA3 is a strictly weaker 3-variable
equation, that is still stronger than the OM law.  OMLs in which
OA3 or OA4 hold are called \textcolor{red}{3OA}s,
\textcolor{red}{4OA}s respectively.

\end{slide}

\begin{slide}

\textcolor{blue}{\textbf{Generalization of Orthoarguesian Law (Definitions)}}
\begin{eqnarray}
\textcolor{red}{a_1{\buildrel (3)\over\equiv}a_2}
&{\buildrel\rm def\over =}& a_1{\buildrel a_3\over\equiv}a_2\  \\
\textcolor{red}{a_1{\buildrel (4)\over\equiv}a_2}
&{\buildrel\rm def\over =}& a_1{\buildrel a_4,a_3\over\equiv}a_2\ \\
\textcolor{red}{a_1{\buildrel (5)\over\equiv}a_2}
&{\buildrel\rm def\over =}& (a_1{\buildrel (4)\over\equiv}a_2)\cup
((a_1{\buildrel (4)\over\equiv}a_5)\cap
(a_2{\buildrel (4)\over\equiv}a_5))\label{5oaoper}\\
\textcolor{red}{a_1{\buildrel (n)\over\equiv}a_2}
&{\buildrel\rm def\over =}& (a_1{\buildrel (n-1)\over\equiv}a_2)\cup
((a_1{\buildrel (n-1)\over\equiv}a_n)\cap
(a_2{\buildrel (n-1)\over\equiv}a_n)),\nonumber\\
& & n\ge 4
\end{eqnarray}
\end{slide}

\begin{slide}
\textcolor{blue}{\textbf{Generalization of Orthoarguesian Law
 (Definitions, cont.)}}

To obtain
${\buildrel (n)\over\equiv}$ we substitute in each
${\buildrel (n-1)\over\equiv}$ subexpression only the two explicit
variables, leaving the other variables the same.  For example,
$(a_2{\buildrel (4)\over\equiv}a_5)$ in (\ref{5oaoper}) means
$(a_2{\buildrel (3)\over\equiv}a_5)\cup ((a_2{\buildrel
(3)\over\equiv}a_4)\cap (a_5{\buildrel (3)\over\equiv}a_4))$ which means
$(((a_2\to_1 a_3)\cap(a_5\to_1 a_3)) \cup((a_2'\to_1 a_3)\cap(a_5'\to_1
a_3)))\cup
(
(((a_2\to_1 a_3)\cap(a_4\to_1 a_3))
\cup((a_2'\to_1 a_3)\cap(a_4'\to_1 a_3)))
\cap
(((a_5\to_1 a_3)\cap(a_4\to_1 a_3))
\cup((a_5'\to_1 a_3)\cap(a_4'\to_1 a_3)))
)
$
\end{slide}

\begin{slide}
\textcolor{blue}{\textbf{Generalization of Orthoarguesian Law
 (cont.)}}

Theorem [Megill/Pavi\v ci\'c, 2000]: The \textcolor{red}{$n$OA {\em laws}}
\begin{eqnarray}
(a_1\to_1 a_3) \cap (a_1{\buildrel (n)\over\equiv}a_2)
\le a_2\to_1 a_3\,.\label{eq:noa}
\end{eqnarray}
hold in   ${\cal C}({\cal H})$.  In addition, they form a series of
successively stronger laws than 3OA and 4OA (proved for $n=5$ and $n=6$;
open problem for $n>6$).

The independence proof for $n=6$ required 10 CPU years on a 192-CPU Linux
cluster at Australian National University.
\end{slide}

\begin{slide}

\textcolor{blue}{\textbf{``Orthoarguesian Identity'' Laws}}

The relations
\begin{eqnarray}
a{\buildrel c\over\equiv}b=1\qquad & \Leftrightarrow & \qquad
a\to_1c=b\to_1c\qquad\mbox{(\textcolor{red}{OI3})}\\
a{\buildrel c,d\over\equiv}b=1\qquad & \Leftrightarrow & \qquad
a\to_1d=b\to_1d\qquad\mbox{(\textcolor{red}{OI4})}
\end{eqnarray}
hold in all 3OAs, 4OAs respectively.

Open problems:\\
\textcolor{red}{OI3 {\em conjecture}}:  All OMLs in which OI3 holds
are 3OAs.\\
\textcolor{red}{OI4 {\em conjecture}}:  All OMLs in which OI4 holds
are 4OAs.

\end{slide}

\begin{slide}
\textcolor{blue}{\textbf{\LARGE \$100 Prize}}

I have ``wasted'' so much time and effort over the past 3 years trying
to prove or disprove the OI3 conjecture that, in an effort to maintain my
sanity, I hereby offer this prize to anyone who proves or disproves
it.

\end{slide}


\begin{slide}
The following equation, if it holds in all OMLs, will prove the
OI3 conjecture (note that $-a$ means $a'$):

$((((-(-a \cup (a \cap c)) \cup ((-(-a \cup (a \cap c)) \cup -(-b
\cup (b \cap c))) \cap (-a \cup -b))) \cup (((-a \cup (a \cap c))
\cap (((-a \cup (a \cap c)) \cap (-b \cup (b \cap c))) \cup (a \cap
b))) \cap c)) \cap ((-(-b \cup (b \cap c)) \cup ((-(-a \cup (a \cap
c)) \cup -(-b \cup (b \cap c))) \cap (-a \cup -b))) \cup (((-b
\cup (b \cap c)) \cap (((-a \cup (a \cap c)) \cap (-b \cup (b \cap
c))) \cup (a \cap b))) \cap c))) \cup ((-a \cup (a \cap c)) \cap (-b
\cup (b \cap c)))) = 1$

Using the Sasaki implication, we can abreviate this as follows:

$(((((a \to_1 c) \cap (((a \to_1 c) \cap (b \to_1 c)) \cup (a \cap b)))
\to_1 c) \cap (((b \to_1 c) \cap (((a \to_1 c) \cap (b \to_1 c)) \cup (a
\cap b))) \to_1 c)) \cup ((a \to_1 c) \cap (b \to_1 c))) = 1$


\end{slide}

\begin{slide}

%
%   (aIc)^(a=c=b) C bIc                   (a) (proved 3OA-equivalent in
%                                             Theorem 1 of email of 28-Apr-02)
%   (aIc)^(a=c=b) C (bIc)^(a=c=b)         (b)
%   -(-aIc)^(a=c=b) =< bIc                (c)
%   -(-aIc)^(a=c=b) C bIc                 (d)
%   -(-aIc)^(a=c=b) C (bIc)^(a=c=b)       (e)
%   c^(aIc)^(a=c=b) =< (bIc)              (f)
%   c^(aIc)^(a=c=b) C (bIc)               (g)
%   c^(aIc)^(a=c=b) C (bIc)^(a=c=b)       (h)
%   ((aIc)^(a=c=b))Ic = ((bIc)^(a=c=b))Ic (i) (proved 3OA-equivalent in
%                                             email of 12-May-02)
%   ((aIc)^(a=c=b))Ic C ((bIc)^(a=c=b))Ic (j)
%
% From earlier emails, we know:
%
%   (a) => (b)
%   (a) => (c) => (d) => (e)
%   (a) => (f) => (g) => (h)
%   (a) <=> (i) => (j)
%
% Here is one minor improvement I have made.  I will prove that
% (h) => (f), establishing
%
%   (f) <=> (g) <=> (h)
\textcolor{blue}{\textbf{Equations Related to the OI3 Conjecture}}
\begin{eqnarray}
(a\to_1 c)\cap (a{\buildrel c\over\equiv}b)  & {\rm C} &  b\to_1 c                  \label{eqa}\\
(a\to_1 c)\cap (a{\buildrel c\over\equiv}b)  & {\rm C} &  (b\to_1 c)\cap (a{\buildrel c\over\equiv}b)    \label{eqb}\\
(a'\to_1 c)'\cap (a{\buildrel c\over\equiv}b)  & \le &  b\to_1 c                    \label{eqc}\\
(a'\to_1 c)'\cap (a{\buildrel c\over\equiv}b)  & {\rm C} &  b\to_1 c                \label{eqd}\\
(a'\to_1 c)'\cap (a{\buildrel c\over\equiv}b)  & {\rm C} &  (b\to_1 c)\cap (a{\buildrel c\over\equiv}b)  \label{eqe}\\
c\cap (a\to_1 c)\cap (a{\buildrel c\over\equiv}b)  & \le &  (b\to_1 c)              \label{eqf}\\
c\cap (a\to_1 c)\cap (a{\buildrel c\over\equiv}b)  & {\rm C} &  (b\to_1 c)          \label{eqg}\\
c\cap (a\to_1 c)\cap (a{\buildrel c\over\equiv}b)  & {\rm C} &  (b\to_1 c)\cap (a{\buildrel c\over\equiv}b)     \label{eqh}\\
((a\to_1 c)\cap (a{\buildrel c\over\equiv}b))\to_1 c & = & ((b\to_1 c)\cap (a{\buildrel c\over\equiv}b))\to_1 c \label{eqi}\\
((a\to_1 c)\cap (a{\buildrel c\over\equiv}b))\to_1 c  & {\rm C} &  ((b\to_1 c)\cap (a{\buildrel c\over\equiv}b))\to_1 c
  \label{eqj}
\end{eqnarray}


\end{slide}


\begin{slide}
\textcolor{blue}{\textbf{Equations Related to the OI3 Conjecture (cont.)}}

All of these equations are implied by the 3OA law.  All of
these equations imply OI3.  Unknown is whether most of them are
equivalent to the 3OA law.  A proof of the OI3 conjecture would
establish all of them as equivalent to the 3OA law.  Known results
are as follows
(note that $\Rightarrow$ means ``can be proved from the axiom
system of OML $+$ the left-hand side equation added as an axiom''):

\qquad   OA3 $\Leftrightarrow$ \ref{eqa} $\Rightarrow$ \ref{eqb}
   $\Rightarrow$ OI3\\
\qquad   OA3 $\Rightarrow$ \ref{eqc} $\Rightarrow$ \ref{eqd}
    $\Rightarrow$ \ref{eqe}  $\Rightarrow$ OI3 \\
\qquad   OA3 $\Rightarrow$ \ref{eqf} $\Leftrightarrow$
    \ref{eqg} $\Leftrightarrow$ \ref{eqh}  $\Rightarrow$ OI3\\
\qquad   OA3 $\Leftrightarrow$ \ref{eqi}
  $\Rightarrow$ \ref{eqj}  $\Rightarrow$ OI3

\end{slide}

\begin{slide}

\begin{center}
\textcolor{blue}{\textbf{References}}

Most of the references for this material can be found at:

\textcolor{darkgreen}{
\texttt{http://us.metamath.org/qlegif/mmql.html{\char`\#}ref}}

\vspace{3ex}

More miscellaneous stuff can be found at:\\
\textcolor{darkgreen}{
\texttt{http://users.shore.net/{\char`\~}ndm/award2003.html}}

\end{center}
\end{slide}

\end{document}


