% The following warnings occur throughout when I compile with pdflatex.
% They don't seem to affect the output significantly.
%
%
% LaTeX Font Warning: Font shape `U/msa/m/n' in size <19.907> not available
% (Font)              size <20.74> substituted on input line 39.
%
% ! pdfTeX warning (ext4): destination with the same identifier (name{page.0}) ha
% s been already used, duplicate ignored
% <to be read again>
%                    \penalty
% l.802 \end{slide}
%                   [26] (./megillaward2005.aux)
%
% LaTeX Font Warning: Size substitutions with differences
% (Font)              up to 0.83301pt have occurred.
%
% LaTeX Warning: There were multiply-defined labels.




\documentclass{slides}

%page size
\usepackage{anysize}
\papersize{8in}{11in}
%left,right,top,bottom
%\marginsize{0.4706in}{0.4706in}{0.476in}{0.4706in}
\marginsize{1in}{1in}{0.2in}{1in}

\usepackage{color}
%\definecolor{mint}{rgb}{.933,1,.98}
\definecolor{mint}{rgb}{.960784,.996078,.976470}
\definecolor{darkgreen}{rgb}{0,.7,0}
\definecolor{lightgray}{gray}{0.75}
\definecolor{mscolor}{rgb}{1.0,0,0}
\definecolor{orange}{rgb}{1.0,0.5,0}

\usepackage{amssymb}
\usepackage{latexsym}
\usepackage{amsthm}

% http://www.misojiro.t.u-tokyo.ac.jp/~kuroky/suribt/hyperref_options.pdf
\usepackage[bookmarks=false]{hyperref}

\begin{document}
\raggedright
\pagecolor{mint}


%%%%%%%%%%%%%%%%%%%%%%%%%%%%%%%%%%%%%%%%%%%%%%%%%%%%%%%%%%%%%%%%%%%%
\begin{slide}

\begin{center}
\textcolor{blue}{\textbf{\LARGE Emulating Hilbert's Epsilon in ZFC}}\\
\vspace{3ex}
{\large Norman Megill}\\
\vspace{1ex}
\textcolor{darkgreen}{\texttt{nm{}@{}alum.mit.edu\qquad
\url{http://metamath.org}}}\\
\vspace{1ex}
August 12, 2005
\end{center}

\end{slide}
%%%%%%%%%%%%%%%%%%%%%%%%%%%%%%%%%%%%%%%%%%%%%%%%%%%%%%%%%%%%%%%%%%%%


%%%%%%%%%%%%%%%%%%%%%%%%%%%%%%%%%%%%%%%%%%%%%%%%%%%%%%%%%%%%%%%%%%%%
\begin{slide}

\begin{center}
\textcolor{blue}{\textbf{Hilbert's epsilon calculus}}
\end{center}


Hilbert's epsilon calculus is described at
\url{http://plato.stanford.edu/entries/epsilon-calculus/}.
The term ``\textcolor{red}{$\varepsilon x \varphi$}'' denotes ``some $x$ satisfying
wff $\varphi$.''

The {\em Transfinite Axiom} is the basic axiom
needed for the epsilon calculus:
\begin{eqnarray}
\varphi \to [\varepsilon x \varphi / x] \varphi
\end{eqnarray}
where $x$ is free in $\varphi$ and $[A / x]\varphi$ denotes
the proper substitution of class-term $A$ for $x$ in $\varphi$.

\end{slide}
%%%%%%%%%%%%%%%%%%%%%%%%%%%%%%%%%%%%%%%%%%%%%%%%%%%%%%%%%%%%%%%%%%%%

%%%%%%%%%%%%%%%%%%%%%%%%%%%%%%%%%%%%%%%%%%%%%%%%%%%%%%%%%%%%%%%%%%%%
\begin{slide}

\begin{center}
\textcolor{blue}{\textbf{Motivation}}
\end{center}

Theorem provers such as HOL use the epsilon calculus extensively as a
proving tool.  Our goal is to be able to translate such proofs
into a form that can be verified by a ZFC-only proof verifier.

Discussion: \url{http://ghilbert.org/choice.txt}

While the Transfinite Axiom represents a form of the Axiom of Choice,
ZFC cannot express it directly.  ZFC can, however, prove the same
epsilon-free theorems as the epsilon calculus.  {\bf We will show a practical
algorithm that can translate an epsilon-calculus proof
(of an epsilon-free theorem) to a ZFC-only
proof.}

\end{slide}
%%%%%%%%%%%%%%%%%%%%%%%%%%%%%%%%%%%%%%%%%%%%%%%%%%%%%%%%%%%%%%%%%%%%

%%%%%%%%%%%%%%%%%%%%%%%%%%%%%%%%%%%%%%%%%%%%%%%%%%%%%%%%%%%%%%%%%%%%
\begin{slide}

\begin{center}
\textcolor{blue}{\textbf{The trivial case of Hilbert's epsilon}}
\end{center}

If there is exactly one element such that a property $\varphi$
 is true,
we can express ``the (unique) element such that
       $\varphi$'' (usually called ``iota'')
 as ``$\bigcup \{ x | \varphi \}$,'' which
emulates Hilbert's epsilon.  Hilbert's Transfinite
Axiom can be easily emulated using this ZFC theorem:
\begin{eqnarray}  %reuuni4
 \exists{!} x \varphi \rightarrow [ \bigcup \{ x | \varphi
\} / x ] \varphi   \label{reuuni4}
\end{eqnarray}

To use it, just detach $\exists{!} x \varphi$ and add the
antecedent $\varphi$ to obtain the Transfinite Axiom instance.
So, assuming $\exists{!} x \varphi$,
\begin{eqnarray}
  \varphi \rightarrow [ \bigcup \{ x | \varphi
\} / x ] \varphi
\end{eqnarray}

\end{slide}
%%%%%%%%%%%%%%%%%%%%%%%%%%%%%%%%%%%%%%%%%%%%%%%%%%%%%%%%%%%%%%%%%%%%

%%%%%%%%%%%%%%%%%%%%%%%%%%%%%%%%%%%%%%%%%%%%%%%%%%%%%%%%%%%%%%%%%%%%
\begin{slide}

\begin{center}
\textcolor{blue}{\textbf{The ZFC axioms}}
\end{center}
\begin{eqnarray}
\mbox{(Ext)}\ \forall z ( z \in x \leftrightarrow z \in y ) \rightarrow x = y & &
\label{ax-ext}
\end{eqnarray}
\begin{eqnarray}
\mbox{(Rep)}\ \forall w \exists y \forall z ( \forall y \varphi \rightarrow z = y )
\rightarrow \exists y \forall z ( z \in y \leftrightarrow \exists w ( w \in x
\wedge \forall y \varphi ) ) & & \ \
\label{ax-rep}
\end{eqnarray}
\begin{eqnarray}
\mbox{(Un)}\ \exists y \forall z ( \exists w ( z \in w \wedge w \in x ) \rightarrow z
\in y)  & &
\label{ax-un}
\end{eqnarray}
\begin{eqnarray}
\mbox{(Pow)}\ \exists y \forall z ( \forall w ( w \in z \rightarrow w \in x )
\rightarrow z \in y)  & &
\label{ax-pow}
\end{eqnarray}
\begin{eqnarray}
\mbox{(Reg)}\ \exists y \,y \in x \rightarrow \exists y ( y \in x \wedge \forall z (
z \in y \rightarrow \lnot z \in x ) )   & &
\label{ax-reg}
\end{eqnarray}
\begin{eqnarray}
\mbox{(Inf)}\ \exists y ( x \in y \wedge \forall z ( z \in y \rightarrow \exists w ( z
\in w \wedge w \in y ) ) )  & &
\label{ax-inf}
\end{eqnarray}
\begin{eqnarray}
\lefteqn{\mbox{(AC)}\ \exists y \forall z \forall w ( ( z \in w
\wedge w \in x ) \rightarrow} \nonumber \\
& & \exists v \forall u ( \exists t ( ( u \in w \wedge w \in t )
 \wedge ( u \in t
\wedge t \in y ) ) \leftrightarrow u = v ))
\label{ax-ac}
\end{eqnarray}


\end{slide}
%%%%%%%%%%%%%%%%%%%%%%%%%%%%%%%%%%%%%%%%%%%%%%%%%%%%%%%%%%%%%%%%%%%%

%%%%%%%%%%%%%%%%%%%%%%%%%%%%%%%%%%%%%%%%%%%%%%%%%%%%%%%%%%%%%%%%%%%%
\begin{slide}

\begin{center}
\textcolor{blue}{\textbf{Just for fun}}
\end{center}

A very short version of the Axiom of Infinity, using only elementary
symbols ($\subset$ is proper subset):
\begin{eqnarray}
\exists x \,x \subset \bigcup x \label{inf5}
\end{eqnarray}

If we allow restricted quantifiers and $\exists !$, the Axiom of
Choice with only
one propositional connective:
\begin{eqnarray}
\exists y \forall z \in x \forall w \in z \exists{!} v \in z \exists u
\in y ( z \in u \wedge v \in u ) \label{ac2}
\end{eqnarray}


\end{slide}
%%%%%%%%%%%%%%%%%%%%%%%%%%%%%%%%%%%%%%%%%%%%%%%%%%%%%%%%%%%%%%%%%%%%

%%%%%%%%%%%%%%%%%%%%%%%%%%%%%%%%%%%%%%%%%%%%%%%%%%%%%%%%%%%%%%%%%%%%
\begin{slide}

\begin{center}
\textcolor{blue}{\textbf{Definitions for set theory (1 of 5)}}
\end{center}

We assume you know:  virtual classes,
subset, power class ${\cal P} x$, empty set
$\varnothing$, universe $V$, unordered and ordered pairs, class builder,
union and intersection (small and big), Cartesian (cross) product,
binary relations.

Capital letters $A$, $B$, $F$, $R$ are variables ranging
over classes (which may be proper).
Small letters $x$, $y$, $z$, $w$, $f$, $g$, etc. range over sets and are the
individual variables of the first-order logic.

Define ``$R$ is a founded relation on (possibly proper) class $A$.''
\begin{eqnarray}
\textcolor{red}{R \,{\rm Fr}\, A} & {\buildrel\rm def\over \leftrightarrow} & \forall x ( ( x \subseteq A \wedge \lnot
x = \varnothing ) \rightarrow \exists y \in x \forall z \in x \lnot z \,R \,y
)   \label{df-fr}
\end{eqnarray}

\end{slide}
%%%%%%%%%%%%%%%%%%%%%%%%%%%%%%%%%%%%%%%%%%%%%%%%%%%%%%%%%%%%%%%%%%%%

%%%%%%%%%%%%%%%%%%%%%%%%%%%%%%%%%%%%%%%%%%%%%%%%%%%%%%%%%%%%%%%%%%%%
\begin{slide}

\begin{center}
\textcolor{blue}{\textbf{Definitions for set theory (2 of 5)}}
\end{center}

Define ``$R$ well-orders $A$.''
\begin{eqnarray}
\textcolor{red}{ R \,{\rm We}\, A} & {\buildrel\rm def\over \leftrightarrow} &
( R \,{\rm Fr}\, A \wedge \forall x \in A
\forall y \in A ( x \,R \,y \vee x = y \vee y \,R \,x ) )\ \ \ \label{dfwe2}
\end{eqnarray}
Define ``$A$ is a transitive class.''
\begin{eqnarray}
\textcolor{red}{ {\rm Tr}\, A} &
 {\buildrel\rm def\over \leftrightarrow} & \bigcup A \subseteq A  \label{df-tr}
\end{eqnarray}
Define the epsilon relation.
\begin{eqnarray}
\textcolor{red}{E} & {\buildrel\rm def\over =} &  \{ \langle x , y \rangle | x \in y \}
\label{df-eprel}
\end{eqnarray}
Define ``$A$ is an ordinal class.''
\begin{eqnarray}
\textcolor{red}{ {\rm Ord}\, A} & {\buildrel\rm def\over \leftrightarrow} &
 {\rm Tr}\, A \wedge E\, {\rm We}\, A  \label{df-ord}
\end{eqnarray}
Define the class of all ordinals.
\begin{eqnarray}
\textcolor{red}{{\rm On}} & {\buildrel\rm def\over =} &
 \{ x | {\rm Ord}\, x \}  \label{df-on}
\end{eqnarray}

\end{slide}
%%%%%%%%%%%%%%%%%%%%%%%%%%%%%%%%%%%%%%%%%%%%%%%%%%%%%%%%%%%%%%%%%%%%

%%%%%%%%%%%%%%%%%%%%%%%%%%%%%%%%%%%%%%%%%%%%%%%%%%%%%%%%%%%%%%%%%%%%
\begin{slide}

\begin{center}
\textcolor{blue}{\textbf{Definitions for set theory (3 of 5)}}
\end{center}

Define ``$A$ is a limit ordinal.''
\begin{eqnarray}
\textcolor{red}{ {\rm Lim}\, A} & {\buildrel\rm def\over \leftrightarrow} &
 {\rm Ord} A \wedge \lnot A = \varnothing
\wedge A = \bigcup A  \label{df-lim}
\end{eqnarray}
Define the successor of a class $A$.
\begin{eqnarray}
\textcolor{red}{{\rm suc}\, A} & {\buildrel\rm def\over =} &
  A \cup \{ A \}  \label{df-suc}
\end{eqnarray}
Define the domain of a class.
\begin{eqnarray}
\textcolor{red}{{\rm dom}\, A} & {\buildrel\rm def\over =} &
  \{ x | \exists y \,x \,A \,y \} \label{df-dm}
\end{eqnarray}
Define the range of a class.
\begin{eqnarray}
\textcolor{red}{{\rm ran}\, A} & {\buildrel\rm def\over =} &  \{ y | \exists x
 \,x \,A \,y \}  \label{dfrn2}
\end{eqnarray}
Define the restriction of a class.
\begin{eqnarray}
\textcolor{red}{( A \restriction B )} & {\buildrel\rm def\over =} &
   A \cap ( B \times V )   \label{df-res}
\end{eqnarray}

\end{slide}
%%%%%%%%%%%%%%%%%%%%%%%%%%%%%%%%%%%%%%%%%%%%%%%%%%%%%%%%%%%%%%%%%%%%

%%%%%%%%%%%%%%%%%%%%%%%%%%%%%%%%%%%%%%%%%%%%%%%%%%%%%%%%%%%%%%%%%%%%
\begin{slide}

\begin{center}
\textcolor{blue}{\textbf{Definitions for set theory (4 of 5)}}
\end{center}

Define the image of a class.
\begin{eqnarray}
\textcolor{red}{( A `` B )} & {\buildrel\rm def\over =} &
 {\rm ran} ( A \restriction B ) \label{df-ima}
\end{eqnarray}
Define the value of a function.  (Applies to any class $F$).
\begin{eqnarray}
\textcolor{red}{( F ` A )} & {\buildrel\rm def\over =} &
 \bigcup \{ x | ( F `` \{ A \} ) = \{ x \} \} \label{df-fv}
\end{eqnarray}
Define ``$A$ is a relation.''
\begin{eqnarray}
\textcolor{red}{ {\rm Rel}\, A} & {\buildrel\rm def\over \leftrightarrow} &
 A \subseteq ( V \times V ) \label{df-rel}
\end{eqnarray}
Define ``class $A$ is a function.''
\begin{eqnarray}
\textcolor{red}{ {\rm Fun}\, A} & {\buildrel\rm def\over \leftrightarrow} &
 {\rm Rel}\, A \wedge \forall x \exists z
\forall y ( x \,A \,y \rightarrow y = z )  \label{dffun3}
\end{eqnarray}
Define ``class $A$ is a function with domain $B$.''
\begin{eqnarray}
\textcolor{red}{ A \,{\rm Fn}\, B} & {\buildrel\rm def\over \leftrightarrow} &
  {\rm Fun}\, A \wedge {\rm dom}\, A = B   \label{df-fn}
\end{eqnarray}

\end{slide}
%%%%%%%%%%%%%%%%%%%%%%%%%%%%%%%%%%%%%%%%%%%%%%%%%%%%%%%%%%%%%%%%%%%%

%%%%%%%%%%%%%%%%%%%%%%%%%%%%%%%%%%%%%%%%%%%%%%%%%%%%%%%%%%%%%%%%%%%%
\begin{slide}

\begin{center}
\textcolor{blue}{\textbf{Definitions for set theory (5 of 5)}}
\end{center}

Define a recursive definition generator on ${\rm On}$
 with characteristic function $F$ and initial value $A$.
\begin{eqnarray}
\lefteqn{\textcolor{red}{{\rm rec} ( F , A )} \ \  {\buildrel\rm def\over =}
 \ \ \bigcup \{ f | \exists x \in {\rm On} ( f \,{\rm Fn}\,
x \wedge \forall y \in x ( f ` y ) =} \nonumber \\
& & ( \{ \langle g , z \rangle | ( ( g =   \varnothing \wedge z = A ) \nonumber \\
& & \vee ( \lnot ( g = \varnothing  \vee {\rm Lim}\,  {\rm dom}\, g )
\wedge \nonumber z = ( F ` ( g ` \bigcup {\rm dom}\,
 g ) ) )  \\
& &  \vee ( {\rm Lim}\,  {\rm dom}\, g \wedge z = \bigcup {\rm ran}\,
 g ) ) \} ` ( f \restriction y ) ) ) \}  \label{dfrdg2}
\end{eqnarray}
Define the cumulative hierarchy of sets function $R_1$.
\begin{eqnarray}
\textcolor{red}{R_1} & {\buildrel\rm def\over =} &  {\rm rec} ( \{ \langle x , y \rangle | y = {\cal P} x \} ,
\varnothing ) \label{df-r1}
\end{eqnarray}
Define the rank function.
\begin{eqnarray}
\textcolor{red}{{\rm rank}} & {\buildrel\rm def\over =} &
\{ \langle x , y \rangle | y = \bigcap \{ z \in {\rm On} |
x \in ( R_1 ` {\rm suc}\, z ) \} \} \label{df-rank}
\end{eqnarray}


\end{slide}
%%%%%%%%%%%%%%%%%%%%%%%%%%%%%%%%%%%%%%%%%%%%%%%%%%%%%%%%%%%%%%%%%%%%

%%%%%%%%%%%%%%%%%%%%%%%%%%%%%%%%%%%%%%%%%%%%%%%%%%%%%%%%%%%%%%%%%%%%
\begin{slide}


\begin{center}
\textcolor{blue}{\textbf{The Main Theorem!}}
\end{center}

Recall our goal:  we want to emulate Hilbert's epsilon $\varepsilon x
\varphi$.

We define two class variables $A$ and $B$,
where $y$ is not free in $\varphi$:
\begin{eqnarray}
 A & = & \{ x | ( \varphi \wedge \forall y ( [ y / x ] \varphi \rightarrow (
{\rm rank} ` x ) \subseteq ( {\rm rank} ` y ) ) ) \} \\
B & = & \bigcup \{ x \in A | \forall y \in A \,\lnot y \,R \,x \}
\end{eqnarray}
Then the following theorem of ZFC emulates Hilbert's Transfinite
Axiom, with the additional antecedent ``$R \,{\rm We}\, A$'':
\begin{eqnarray}
%\mbox{(HTA)} \qquad
  R \,\,{\rm We}\,\, A \rightarrow ( \varphi \rightarrow [ B / x ] \varphi )
    \label{hta}
\end{eqnarray}
Class $B$ emulates Hilbert's epsilon!

\small{(Note: In English,
$A$ is the collection of all sets of minimum rank
with property $\varphi$.  $B$ is the smallest member of $A$ w.r.t.
some well-ordering relation $R$.)}


\end{slide}
%%%%%%%%%%%%%%%%%%%%%%%%%%%%%%%%%%%%%%%%%%%%%%%%%%%%%%%%%%%%%%%%%%%%

%%%%%%%%%%%%%%%%%%%%%%%%%%%%%%%%%%%%%%%%%%%%%%%%%%%%%%%%%%%%%%%%%%%%
\begin{slide}

\begin{center}
\textcolor{blue}{\textbf{Two key auxilliary theorems}}
\end{center}

{\em Well-ordering theorem} (derived from the Axiom of Choice):  for any
set $x$, there exists a set $y$ s.t. $y$ well-orders $x$.
\begin{eqnarray}
\exists y \,y \,{\rm We}\, x  \label{weth}
\end{eqnarray}


{\em Scott's trick} collects all sets that have a certain property and are of
smallest possible rank.  The following amazing
theorem shows that the resulting
collection exists, i.e. is a set.
\begin{eqnarray}
 \{ x | ( \varphi \wedge \forall y ( [ y / x ] \varphi \rightarrow (
{\rm rank} ` x ) \subseteq ( {\rm rank} ` y ) ) ) \} \in V  \label{scottexs}
\end{eqnarray}
where $y$ is not free in $\varphi$.  In other words, the class $A$
on the previous slide is a set, which is crucial for the well-ordering
theorem to work!


\end{slide}
%%%%%%%%%%%%%%%%%%%%%%%%%%%%%%%%%%%%%%%%%%%%%%%%%%%%%%%%%%%%%%%%%%%%


%%%%%%%%%%%%%%%%%%%%%%%%%%%%%%%%%%%%%%%%%%%%%%%%%%%%%%%%%%%%%%%%%%%%
\begin{slide}

\begin{center}
\textcolor{blue}{\textbf{The algorithm - case 1}}
\end{center}

Suppose the set $A$ in Theorem~\ref{hta} has a constructible well-ordering
(rather than just the existence implied by Theorem~\ref{weth}).
For example,
 $A$ might be a subset of the natural
numbers.  In that case, we
simply substitute the well-ordering in place of $R$ and
 detach $R \,\,{\rm We}\,\, A$.  The result is the necessary
instance of Hilbert's
Transfinite Axiom.  I.e. if we can find an $R$ s.t. we
can prove $R \,\,{\rm We}\,\, A$,
then (from Th.~\ref{hta})
\begin{eqnarray}
%\mbox{(HTA)} \qquad
   \varphi \rightarrow [ B / x ] \varphi
\end{eqnarray}
Note that the trivial case of unique existence, discussed at the
beginning of this talk, is also covered by case 1, although
Theorem~\ref{reuuni4} may be preferred for simplicity.



\end{slide}
%%%%%%%%%%%%%%%%%%%%%%%%%%%%%%%%%%%%%%%%%%%%%%%%%%%%%%%%%%%%%%%%%%%%

%%%%%%%%%%%%%%%%%%%%%%%%%%%%%%%%%%%%%%%%%%%%%%%%%%%%%%%%%%%%%%%%%%%%
\begin{slide}

\begin{center}
\textcolor{blue}{\textbf{The algorithm - case 2}}
\end{center}

Suppose the set $A$ in Theorem~\ref{hta} does not have a constructible
well-ordering.  We substitute a temporary dummy variable, say $w$, for
$R$ in Theorem~\ref{hta}.  In each step in the epsilon-calculus proof
referencing the Transfinite Axiom, we replace the Transfinite Axiom by
Theorem~\ref{hta} with a temporary dummy variable, say $w$, for $R$, and
carry along in the proof the extra antecedent $w\,{\rm We}\, A$ in each
step containing a reference to $B$ (the object that emulates Hilbert's
epsilon).  Note that $B$ will have $w$ as a free variable, so this
antecedent cannot be eliminated.  But since the final theorem is
epsilon-free, at the end we can existentially quantify $w\,{\rm We}\, A$
then detach it with the Well-Ordering Theorem~\ref{weth}.

\end{slide}
%%%%%%%%%%%%%%%%%%%%%%%%%%%%%%%%%%%%%%%%%%%%%%%%%%%%%%%%%%%%%%%%%%%%

\begin{center}
\textcolor{blue}{\textbf{The algorithm - case 2 - continued}}
\end{center}

\begin{tabbing}
\textbf{Epsilon-}\=\textbf{calculus proof}    \=  \hspace{2em} \=
                     \textbf{ZFC} \= \textbf{proof}         \\
    \>  $\vdots$            \>       \>    \>  $\vdots$                  \\
  $ \varphi \to [\varepsilon x \varphi / x] \varphi$  \> \>  \>
              $w \,\,{\rm We}\,\, A \rightarrow ( \varphi
                                  \rightarrow [ B(w) / x ] \varphi )$  \\
    \>  $\vdots$            \>       \>    \>  $\vdots$                  \\
  (manipulate $\varepsilon x \varphi$)  \> \>  \>
              (manipulate $B(w)$)  \\
    \>  $\vdots$            \>       \>    \>  $\vdots$                  \\
  ($\varepsilon x \varphi$-free result)   \> \>  \>
              $w \,\,{\rm We}\,\, A \rightarrow$
                             ($B(w)$-free result)  \\
                                \> \>  \>
              $\exists w \,w \,\,{\rm We}\,\, A \rightarrow$
                             ($B(w)$-free result)  \\
                                \> \>  \>
                             ($B(w)$-free result)  \\
    \>  $\vdots$            \>       \>    \>  $\vdots$                  \\
\end{tabbing}


%%%%%%%%%%%%%%%%%%%%%%%%%%%%%%%%%%%%%%%%%%%%%%%%%%%%%%%%%%%%%%%%%%%%
\begin{slide}


\end{slide}
%%%%%%%%%%%%%%%%%%%%%%%%%%%%%%%%%%%%%%%%%%%%%%%%%%%%%%%%%%%%%%%%%%%%


%%%%%%%%%%%%%%%%%%%%%%%%%%%%%%%%%%%%%%%%%%%%%%%%%%%%%%%%%%%%%%%%%%%%
\begin{slide}

\begin{center}
\textcolor{blue}{\textbf{Appendix - Equation references}}
\end{center}

The following list provides the hyperlinks to the formal proofs
for most of the theorems.

Eq.~\ref{reuuni4}---\url{http://us.metamath.org/mpegif/reuuni4.html}     \\
Eq.~\ref{ax-ext}---\url{http://us.metamath.org/mpegif/ax-ext.html} \\
Eq.~\ref{ax-rep}---\url{http://us.metamath.org/mpegif/ax-rep.html} \\
Eq.~\ref{ax-un}---\url{http://us.metamath.org/mpegif/ax-un.html} \\
Eq.~\ref{ax-pow}---\url{http://us.metamath.org/mpegif/ax-pow.html}
Eq.~\ref{ax-reg}---\url{http://us.metamath.org/mpegif/ax-reg.html} \\
Eq.~\ref{ax-inf}---\url{http://us.metamath.org/mpegif/ax-inf.html} \\
Eq.~\ref{ax-ac}---\url{http://us.metamath.org/mpegif/ax-ac.html} \\
Eq.~\ref{inf5}---\url{http://us.metamath.org/mpegif/inf5.html} \\
Eq.~\ref{ac2}---\url{http://us.metamath.org/mpegif/ac2.html} \\

\end{slide}
%%%%%%%%%%%%%%%%%%%%%%%%%%%%%%%%%%%%%%%%%%%%%%%%%%%%%%%%%%%%%%%%%%%%

%%%%%%%%%%%%%%%%%%%%%%%%%%%%%%%%%%%%%%%%%%%%%%%%%%%%%%%%%%%%%%%%%%%%
\begin{slide}

Eq.~\ref{df-fr}---\url{http://us.metamath.org/mpegif/df-fr.html} \\
Eq.~\ref{dfwe2}---\url{http://us.metamath.org/mpegif/dfwe2.html} \\
Eq.~\ref{df-tr}---\url{http://us.metamath.org/mpegif/df-tr.html} \\
Eq.~\ref{df-eprel}---\url{http://us.metamath.org/mpegif/df-eprel.html} \\
Eq.~\ref{df-ord}---\url{http://us.metamath.org/mpegif/df-ord.html} \\
Eq.~\ref{df-on}---\url{http://us.metamath.org/mpegif/df-on.html} \\
Eq.~\ref{df-lim}---\url{http://us.metamath.org/mpegif/df-lim.html} \\
Eq.~\ref{df-suc}---\url{http://us.metamath.org/mpegif/df-suc.html} \\
Eq.~\ref{df-dm}---\url{http://us.metamath.org/mpegif/df-dm.html} \\
Eq.~\ref{dfrn2}---\url{http://us.metamath.org/mpegif/dfrn2.html} \\
Eq.~\ref{df-res}---\url{http://us.metamath.org/mpegif/df-res.html} \\
Eq.~\ref{df-ima}---\url{http://us.metamath.org/mpegif/df-ima.html} \\
Eq.~\ref{df-fv}---\url{http://us.metamath.org/mpegif/df-fv.html}
Eq.~\ref{df-rel}---\url{http://us.metamath.org/mpegif/df-rel.html} \\
Eq.~\ref{dffun3}---\url{http://us.metamath.org/mpegif/dffun3.html} \\
Eq.~\ref{df-fn}---\url{http://us.metamath.org/mpegif/df-fn.html} \\

\end{slide}
%%%%%%%%%%%%%%%%%%%%%%%%%%%%%%%%%%%%%%%%%%%%%%%%%%%%%%%%%%%%%%%%%%%%

%%%%%%%%%%%%%%%%%%%%%%%%%%%%%%%%%%%%%%%%%%%%%%%%%%%%%%%%%%%%%%%%%%%%
\begin{slide}

Eq.~\ref{dfrdg2}---\url{http://us.metamath.org/mpegif/dfrdg2.html} \\
Eq.~\ref{df-r1}---\url{http://us.metamath.org/mpegif/df-r1.html} \\
Eq.~\ref{df-rank}---\url{http://us.metamath.org/mpegif/df-rank.html} \\
Eq.~\ref{hta}---\url{http://us.metamath.org/mpegif/hta.html} \\
Eq.~\ref{weth}---\url{http://us.metamath.org/mpegif/weth.html}           \\
Eq.~\ref{scottexs}---\url{http://us.metamath.org/mpegif/scottexs.html}


\end{slide}
%%%%%%%%%%%%%%%%%%%%%%%%%%%%%%%%%%%%%%%%%%%%%%%%%%%%%%%%%%%%%%%%%%%%


\end{document}


